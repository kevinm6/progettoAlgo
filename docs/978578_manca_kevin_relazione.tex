\documentclass{article}

\usepackage{listings}
\usepackage{xcolor}
\usepackage{caption}
\usepackage{float}
\usepackage[a4paper,margin=3.5cm]{geometry}
\usepackage{hyperref}
\hypersetup{%
  colorlinks=true,% hyperlinks will be coloured
  linkcolor=black,% hyperlink text will be green
  linkbordercolor=red,% hyperlink border will be red
}

\title{Relazione Progetto Algoritmi\\ (Piastrelle Digitali)}
\author{Kevin Manca, matricola 978578}
\date{10 Luglio 2024}

\begin{document}
\maketitle
\tableofcontents

\section{Introduzione}\label{sec:intro}
Questo documento contiene una panoramica e la documentazione relativa alle strutture dati, gli algoritmi scelti (con i costi a loro associati), come è stato modellato il problema indicato nella specifica, oltre ad opportune scelte implementative.\\
Insieme a questo documento, sono presenti all'interno dell'archivio \texttt{978578\_manca\_kevin.zip}
altri file richiesti dalla specifica:
 
 - File go: \texttt{978578\_manca\_kevin.go} contenente il codice sorgente
 
 - File di test go: \texttt{978578\_manca\_kevin\_test.go} che contiene le funzioni utili a testare il programma

 - Cartella di test: \texttt{test/} che contiene ulteriori sottocartelle con i file utilizzati per il testing del programma


\subsection{Problema}
Il problema imposto dalla specifica è quello di studiare le configurazioni di insiemi di piastrelle digitali su un piano, e la loro influenza sulle piastrelle circonvicine.\\
In questa sezione viene fatta una panoramica relativa alla specifica ed alcuni concetti presenti in questo documento e come sono stati modellati per una possibile soluzione al problema.  

\subsubsection{Piano e Piastrelle}\label{sec:tiles}
Il \textbf{piano} è suddiviso in quadrati di lato unitario, ciascuno dei quali è occupato da una \textbf{piastrella}, la quale può essere \textit{accesa} o \textit{spenta}.
La \textit{piastrella(a, b)} rappresenta la piastrella di lato unitario con vertici $(a, b),\ (a + 1, b),\ (a + 1, b + 1),\ (a, b + 1)$.\\
Ogni piastrella è inoltre caratterizzata da un \textit{colore}, quando una piastrella è accesa appare colorata.\\
Il colore è l'insieme dei colori disponibili sulle stringhe dell'alfabeto.\\
Ogni piastrella può inoltre avere un'intensità differente (indicata con un \textit{numero intero}).\\
Due piastrelle sono dette \textbf{circonvicine} se hanno in comune almeno un vertice.

\subsubsection{Blocco}
Il \textbf{blocco} è una regione massimale di \textit{piastrelle accese}. Il blocco di appartenenza della \textit{Piastrella(x, y)} è il blocco che la contiene.
Definiamo inoltre un \textbf{blocco omogeneo} come un \textit{blocco} nel quale tutte le piastrelle accese hanno il medesimo colore.

\subsubsection{Propagazione del colore}
L’intorno della piastrella \textit{p} è l’insieme di tutte le piastrelle \textit{circonvicine} ad \textit{p} e differenti da \textit{p}.
Possiamo quindi definire \textit{regole di propagazione} del tipo $k_{1} \alpha_{1} + k_{2}\alpha_{2} + \ldots + k_{n}\alpha_{n} \rightarrow \beta$\\
dove n è un intero, $\alpha_{1}, \ldots , \alpha_{n}$ sono stringhe sull’alfabeto ${a, b, \ldots , z}$ tutte differenti tra loro, $\beta$ è una stringa sull’alfabeto ${a, b, . . . , z}$, $k_i$ sono interi positivi la cui somma non supera 8.
Una regola è applicabile a una piastrella \textit{p} se il suo intorno contiene almeno $k_{i}$ piastrelle di colore $\alpha_{i}, \forall i \in \{1, \ldots, n\}$.\\
L'applicazione della regola cambia in $\beta$ il colore della piastrella \textit{p}.
Le regole di propagazione vengono memorizzate in ordine cronologico di inserimento (\textit{FIFO}).\\
È possibile inoltre \textit{propagare i colori} su un blocco: per ogni piastrella colorata \textit{p} che appartiene al blocco, si applica la prima regola applicabile su p rispetto all'intorno di \textit{p} come risulta prima dell'inizio della procedura di propagazione sul blocco.\\
Durante l’esecuzione definiamo \textit{consumo} di una regola la quantità determinata dal numero totale di piastrelle a cui la regola è stata applicata dal momento della sua definizione.

\section{Modellazione}

\subsection*{Generale e blocchi}
Il problema è stato modellato come un \textit{grafo} non orientato; indicando il grafo come $G = (V, E)$ dove:
\begin{itemize}
  \item \textit{V}: l'insieme dei vertici, ovvero una piastrella
  \item \textit{E}: insieme degli archi che in questo caso rappresentano il punto in comune tra le piastrelle
\end{itemize}

Gli \textit{archi} non sono orientati, in quanto è possibile partire da qualsiasi vertice per definire un blocco, e non sono pesati.
\\ \\
I blocchi e blocchi omogenei comportano una visita nel \textit{grafo}, per identificare le piastrelle circonvicine e quindi quelle contenute nel blocco, ma anche per verificarne l'intensità e il loro colore (blocco omogeneo).\\
Una possibile applicazione di questa visita, che poi è quella utilizzata nel progetto, è quella della \textbf{D}epth \textbf{F}irst \textbf{S}earch (o ricerca in profondità) che fissa un vertice origine (ovvero una piastrella), segna come visitato tale vertice, esplora i vertici vicini non visitati e continua in modo ricorsivo fino a quando non trova un vertice che non ne ha altri non visitati.

\subsection*{Propagazione}
Relativamente alla \textit{propagazione di colore} da una \textit{piastrella (x, y)}, è necessario prima verificare la validità delle regole (individualmente), andando quindi a calcolare i valori delle piastrelle circonvicine a quella di partenza e compararli con la regola.\\
Una regola sarà valida e potrà essere applicata se l'intorno di piastrelle circonvicine è sufficiente per eseguirla.\\
Mentre per la \textit{propagazione sul blocco}, è necessario iterare su ogni singola piastrella appartenente al blocco e applicare ad essa la prima \textit{regola valida}.

\subsection*{Pista}\label{sec:pista}
La modellazione scelta per la pista è quella di un \textit{cammino in un grafo}, del quale si è a conoscenza del punto di origine ($piastrella(x_1,\ y_1)$) e di arrivo ($piastrella(x_2,\ y_2)$).\\
La \textit{pista più breve} invece può essere rappresentata come il \textit{cammino minimo nel grafo} tra due vertici (\textit{piastrelle}).\\
Trattandosi di un grafo non pesato, la soluzione ideata è quella di applicare la \textbf{B}readth \textbf{F}irst \textbf{S}earch (o ricerca in ampiezza), che differentemente dalla \textit{DFS} esplora i vertici vicini (adiacenti) prima di continuare l'esplorazione in quelli più lontani, in modo da garantire che il percorso risultante sia quello più breve.

\subsubsection*{Lunghezza}
La \textit{lunghezza} di una \textit{pista} richiesta dalla specifica, rappresenta la distanza tra i vertici e può essere definita (con la modellazione data) come il numero di archi della pista \(+ 1\), quindi il valore del \textit{cammino \(+ 1\)}.

\section{Strutture Dati}
Di seguito sono riportate le \textit{strutture dati} utilizzate e le relative stime dei costi spaziali.
Sono presenti inoltre, nel codice sorgente, ulteriori commenti utili a chiarire alcuni aspetti e scelte adottate.

\subsection*{Piano}
La struttura \texttt{Piano} (modellato come grafo) rappresenta un piano costituito da piastrelle e regole. Utilizzo una mappa (\texttt{map[punto]*Piastrella}) per memorizzare le piastrelle e una slice (\texttt{*[]Regola}) per le regole.
Memorizzo solo le piastrelle \textit{accese}, ciò comporta che quelle \textit{spente} risultano inesistenti e vengono aggiunte alla mappa esclusivamente se vengono accese.

\begin{itemize}
  \item \texttt{piastrelle}: \(O(n)\), \(n\) numero di piastrelle
  \item \texttt{regole}: \(O(m)\), \(m\) numero di regole
\end{itemize}
\textbf{Spazio totale}: \(O(n) + O(m)\)
\\ \\
L'implementazione scelta è dovuta ad alcune considerazioni in merito ad altre rappresentazioni possibili.
Nello specifico:
\begin{itemize}
  \item \textit{matrice di incidenza/adiacenza}: essendo il piano potenzialmente infinito (da specifica), lo spazio occuperebbe \(O(n^2)\) con \textit{n} numero di vertici
  \item \textit{lista di incidenza/adiacenza}: l'operazione di inserimento (eseguita spesso) avrebbe comportato l'utilizzo di \textit{append()} e quindi un tempo pari a \(O(1^*)\)
\end{itemize}

I vantaggi quindi dell'implementazione utilizzata sono una \textit{maggiore efficienza di spazio e tempo}, infatti attraverso le mappe utilizzate le operazioni di \textit{inserimento}, \textit{accesso} e \textit{modifica} sono in tempo costante.

\subsection*{Piastrella}
Rappresenta una singola \texttt{piastrella} con coordinate (\textit{(x int, y int)}), colore (\textit{string}) e intensità (\textit{int}).\\
Le coordinate della piastrella indicano il \textit{vertice di origine} (in basso a sinistra), il colore è semplicemente la stringa sull'alfabeto e l'intensità è un numero intero \(\geq 0\) (\(0\) indica che la piastrella è \textit{spenta}).
\\ \\
\textbf{Spazio totale}: \(O(1)\) per ogni piastrella (\textit{3 interi + 1 stringa})

\subsection*{Punto}
Rappresenta un punto con coordinate (\textit{(x int, y int)}), ovvero un vertice del grafo.\\ \\
\textbf{Spazio totale}: \(O(1)\) per ogni \textit{punto} (tipo \textit{intero})

\subsection*{Regola}\label{subsec:rule}
Rappresenta una \texttt{regola} di propagazione con una stringa completa (\textit{istruzioneCompleta string}), un colore (\textit{string}), un consumo (\textit{uint}) e una mappa dei valori di colore (\textit{map[string]int}).\\
Rispetto alla specifica ho scelto di aggiungere la mappa dei valori (\textit{valColore}) che mi risulta utile nel verificare ad esempio la possibilità di applicare una regola.\\ \\
\textbf{Spazio totale}: \(O(k)\), \(k\) numero di colori nella mappa \texttt{valColore}

\section{Funzioni}\label{sec:funcs}

\subsection{colora}\label{subsec:color}
Colora una piastrella specifica identificata dalle coordinate (\(x, y\)) con un dato colore (\textit{alpha}) e intensità (\textit{i}).\\ \\
\textbf{Tempo}: \(O(1)\) per aggiungere/aggiornare una piastrella nella mappa\\
\textbf{Spazio}: \(O(1)\), aggiunge una singola piastrella alla mappa

\subsection{spegni}
Spegne una piastrella, impostando la sua intensità a zero.\\ \\
\textbf{Tempo}: \(O(1)\) per aggiornare l'intensità della piastrella\\
\textbf{Spazio}: \(O(1)\), non aggiunge nuove piastrelle

\subsection{regola}
Aggiunge una nuova regola (\textit{r}) di propagazione al piano. Questa operazione viene fatta ottenendo prima le istruzioni della regola, iterando su di esse e successivamente aggiungendo la regola alla slice.\\ \\
\textbf{Tempo}: \(O(n)\), \(n\) lunghezza della stringa della regola \textit{r}
\begin{itemize}
  \item \(O(n)\), \textit{strings.Fields(r)} 
  \item \(O(1)\), creazione della mappa dei valori
  \item \(O(t)\), iterazione sulle istruzioni con \textit{t} pari al numero di istruzioni
  \item \(O(1^*)\), append della regola alla slice di regole in tempo ammortizzato (se nel caso peggiore la slice debba essere ridimensionata)
\end{itemize}
\textbf{Spazio}: \(O(t)\), \textit{t} numero di istruzioni della regola

\subsection{stato}
Stampa e restituisce il colore e l’intensità di una piastrella identificata dalle coordinate (\(x,\ y\)).\\
Se la piastrella è \textit{spenta}, non stampa nulla e restituisce la stringa vuota e l'intero \(0\).\\ \\
\textbf{Tempo}: \(O(1)\), accesso diretto alla mappa\\
\textbf{Spazio}: \(O(1)\), non modifica/utilizza strutture dati aggiuntive

\subsection{stampa}
Stampa tutte le regole di propagazione, nell'ordine attuale.\\
Viene effettuata un iterazione sulle regole del \textit{piano} ed effettuo poi una divisione della stringa \textit{istruzioneCompleta} che rappresenta l'intera stringa della regola.\\ \\
\textbf{Tempo}: \(O(n\ \cdot\ k_{max})\), \(n\) numero di regole e \(k_{max}\) lunghezza massima della stringa \textit{istruzioneCompleta} tra tutte le regole\\
\begin{itemize}
  \item \(O(n)\), iterazione su \(n\) numero di regole
  \item \(O(k_i)\), complessità di \textit{strings.SplitN} con \(k_i\) lunghezza della stringa completa della regola i-esima
  \item \(O(1)\), per ogni regola passata a \textit{fmt.Printf}
\end{itemize}
\textbf{Spazio}: \(O(1)\), non alloco strutture dati aggiuntive (che crescerebbero con l'input)

\subsection{blocco e bloccoOmog}\label{sec:block}
Calcola la somma delle intensità delle piastrelle in un \textit{blocco omogeneo} o non omogeneo, utilizzando la \nameref{subsec:dfs} ed gestisce sia il caso di \textit{blocco} che di \textit{blocco omogeneo}.\\
Viene preventivamente verificato che le coordinate passate corrispondano ad una \textit{piastrella accesa} e in caso affermativo si inizializza l'intensità ad valore della piastrella attuale.\\ \\
\textbf{Tempo}: \(O(n\ +\ m)\), la \textit{DFS} \(O(n\ +\ m)\) con \(n\) numero di piastrelle nel blocco e  \(m\) numero di archi; accesso alla mappa e stampa in tempo costante \(O(1)\)\\
\textbf{Spazio}: \(O(n)\) per la mappa delle \textit{visite} nel caso peggiore di \(n\) piastrelle

\subsection{propaga}\label{sec:prop}
Calcola i colori dell'intorno tramite la funzione \nameref{subsec:evalColorsAround}, che vengono poi passati alla funzione \nameref{subsec:checkRule}, la quale restituisce la regola valida da applicare (se esiste), richiamando la funzione "\nameref{subsec:color}" e incrementando il \textit{consumo} di tale \textit{regola}.\\
Se nessuna regola valida viene trovata (la funzione di utilità "\nameref{subsec:checkRule}" restituisce \textit{nil}) e questa funzione non fa niente.
\\ \\
\textbf{Tempo}: \(O(R \cdot C)\) nel caso peggiore, dove \(R\) è il numero di regole e \(C\) è il numero di colori per regola\\
\textbf{Spazio}: \(O(1)\), non vengono utilizzate strutture dati aggiuntive tranne la mappa che comporta comunque una quantità di memoria massima richiesta costante

\subsection{propagaBlocco}\label{sec:propBlock}
Propaga il colore su un blocco di piastrelle dove appartiene la piastrella di coordinate \((x,\ y)\) passate come parametro.\\
Utilizzo la \nameref{subsec:dfs} (\(O(n\ +\ m)\)) per ottenere il \textit{blocco} di appartenenza della piastrella(x, y), itero sulle piastrelle del blocco (\(O(v)\)) e all'interno dell'iterazione, per ogni piastrella, verifico la validità della regola (\(O(r)\))\\ \\
\textbf{Tempo}: \(O(n \cdot r \cdot c)\), \(n\) è il numero di piastrelle nel blocco, \(r\) è il numero di regole, e \(c\) è il numero di colori nella regola\\
\textbf{Spazio}: \(O(n)\) nel caso peggiore tutte le strutture dati di memorizzazione hanno \(O(n)\), sia per la mappa \textit{visite}, la \textit{DFS}, che lo \textit{stack di ricorsione}, pari alla dimensione del \textit{blocco} in questo caso

\subsection{ordina}
Ordina le \textit{regole di propagazione} (specificate nella sezione "\nameref{subsec:rule}") per \textit{consumo}.\\
Quando due regole hanno lo stesso \textit{consumo} mantengo il loro ordine precedente rispetto all'applicazione della funzione \textit{ordina}.\\ \\
\textbf{Tempo}: \(O(n \log n)\), \(n\) numero di regole, dovuto all'algoritmo di ordinamento \href{https://pkg.go.dev/sort#SliceStable}{sliceStable} (si tratta di una versione modificata di \textit{mergeSort})\\
\textbf{Spazio}: \(O(1)\), poiché l'ordinamento avviene in loco, sulla slice di regole da applicare, senza utilizzare spazio aggiuntivo oltre alla memoria già allocata per le \textit{regole} stesse

\subsection{pista}
Stampa la \textit{pista} (un cammino) seguendo una \textit{sequenza di direzioni} a partire da una piastrella (\(x,\ y\)), solo se tale pista è definita, altrimenti non stampa niente.\\
Internamente chiama la funzione "\nameref{subsec:getPista}" (descritta nell'apposita sezione) che segue la \textit{sequenza di direzioni} e aggiorna la \textit{stringa da stampare} se trova le piastrelle.\\
Se la stringa risultante è vuota, significa che nessuna pista è stata trovata e non viene stampato nulla, altrimenti viene stampata la sequenza delle piastrelle, con le relative informazioni come da specifica.\\
\\
\textbf{Tempo}: \(O(k)\), \(k\) lunghezza della \textit{sequenza di direzioni}, stesso costo per la funzione \textit{strings.Split} di \(k\) numero di direzioni e la funzione \nameref{subsec:getPista}\\
\textbf{Spazio}: \(O(k)\), \(k\) lunghezza della \textit{sequenza di direzioni} (è richiesto lo spazio per la memorizzazione della stringa dipendente appunto da esse)

\subsection{lung}
Calcola la lunghezza della pista più breve tra due piastrelle usando la \textit{BFS} (\nameref{subsec:bfs}), in pratica, cerca il \textit{cammino minimo} tra due vertici e ne stampa la \textit{lunghezza} (se definito).\\
La complessità spaziale e temporale quindi di questa funzione sono strettamente influenzate dalla funzione \textit{calcolaPistaBreve} (e quindi la BFS).\\
\textbf{Tempo}: \(O(V\ +\ E)\), con \(V\) numero di piastrelle del piano da esplorare, \(E\) numero di archi (vertici in comune delle piastrelle), pari quindi alla \textit{BFS}\\
\textbf{Spazio}: \(O(n)\) per le strutture dati di BFS, con \(n\) numero di vertici

\section{Funzioni di utilità}
In questa sezione sono presenti le \textit{funzioni} utili alle altre \nameref{sec:funcs} principali che permettono di evitare la ridondanza di codice e di raggruppare comportamenti condivisi richiesti.\\

\subsection{calcolaDeltaVertice}
Calcola e restituisce un \textit{punto} dati in input il \textit{punto di origine} e lo spostamento sulla coordinata \(x\) (\textit{deltaX int}) e \(y\) (\textit{deltaY int}).\\ \\
\textbf{Tempo}: \(O(1)\), operazione aritmetica basilare in tempo costante \\
\textbf{Spazio}: \(O(1)\), restituisce un \textit{punto} che come descritto precedentemente è una struttura composta da due interi (\(x,\ y\))

\subsection{dfs}\label{subsec:dfs}
La visita in profondità (\textit{DFS}) viene usata per le funzioni \nameref{sec:block} e \nameref{sec:propBlock} per ottenere le piastrelle di un blocco partendo da una piastrella indicata dal vertice (\textit{punto} di coordinate \(x,\ y\)).\\
La ricerca procederà ricorsivamente se la \textit{piastrella} (vertice) attuale non è stata ancora visitata ed è \textit{accesa}.\\
Ho definito questa funzione utilizzando ulteriori parametri per renderla versatile per ulteriori calcoli.\\
Sono presenti infatti un \textit{booleano omogeneo}, una \textit{mappa blocco} ed un \textit{puntatore ad intero sum}.

\begin{itemize}
  \item \textbf{omogeneo} \textit{bool}: \textit{true} per calcolare appunto le piastrelle circonvicine nel caso di un \textit{blocco omogeneo}, \textit{false} altrimenti
  \item \textbf{blocco} \textit{map[punto]*Piastrella}: conterrà un puntatore alla mappa (le mappe in Go sono passate per riferimento) nel caso di \textit{blocco omogeneo}, altrimenti sarà \textit{nil} e non verrà modificato
  \item \textbf{sum} \textit{*int}: nel caso di \textit{blocco/blocco omogeneo}, conterrà l'indirizzo di un intero inizializzato nella funzione chiamante (\textit{blocco()}), questo per sommare il valore dell'intensità delle piastrelle in loco ed evitare di restituire valori; sarà \textit{nil} se non è richiesta la somma come nella funzione \nameref{sec:propBlock}
\end{itemize}

\textbf{Tempo}: \(O(V\ +\ E)\), \(V\) numero di piastrelle nel blocco, \(E\) numero di archi
\begin{itemize}
  \item \(O(1)\): aggiunta vertice alla mappa
  \item \(O(1)\): per iterazione sulle direzioni possibili (8 nel caso del progetto)
  \item \(O(1)\): \textit{calcolaDeltaVertice} come riportato in precedenza
  \item \(O(1)\): verifica delle condizioni e ricorsione
\end{itemize}
\textbf{Spazio}: \(O(n)\) per la mappa delle visite, \(n\) numero di vertici visitati
\begin{itemize}
  \item \(O(n)\): memorizzazione del booleano per ogni piastrella (vertice)
  \item \(O(n)\): caso peggiore, la mappa può richiedere \(O(n)\) piastrelle
  \item \(O(n)\): stack della chiamata ricorsiva, pari al numero di piastrelle \(n\)
  \item \(O(1)\): spazio della variabile \textit{sum} 
\end{itemize}

\subsection{piastrelleCirconvicine}
Restituisce le piastrelle circonvicine (definite nella sezione "\nameref{sec:tiles}") come una mappa (\textit{map[punto]*Piastrella}), partendo da un vertice (\textit{punto}) di origine, seguendo le 8 direzioni possibili.
\\ \\
\textbf{Tempo}: \(O(1)\), \textit{for-range} esegue 8 iterazioni (sono 8 le direzioni), mantenendo costante l'accesso alla mappa ed eventualmente aggiungendo un elemento a quest'ultima (in \(O(1)\))\\
\textbf{Spazio}: \(O(1)\), la mappa \textit{vicine} contiene al massimo 8 elementi

\subsection{verificaRegola}\label{subsec:checkRule}
Verifica che i colori della mappa dell'intorno passato come argomento (\textit{coloriIntorno map[string]int}), siano sufficienti a soddisfare quelli della regola in esame.\\
In caso affermativo, restituisce tale regola (un puntatore ad essa), altrimenti restituisce \textit{nil}.
\\ \\
\textbf{Tempo}: \(O(c)\), \(c\) pari al numero di colori distinti, che essendo limitato cresce in modo lineare rispetto al suo numero\\
\textbf{Spazio}: \(O(1)\), la mappa \textit{valoriColore} contiene al massimo 8 colori, pari a tutte le piastrelle circonvicine di colore, diverso garantendo l'utilizzo di spazio aggiuntivo costante

\subsection{calcolaPista}\label{subsec:getPista}
Calcola la \textit{pista} (definita nella sezione "\nameref{sec:pista}"), seguendo la sequenza di direzioni passate come parametro ed aggiorna la \textit{stringa da stampare} (come richiesto da specifica) se tale pista è definita. Se la pista non esiste, non restituisce nulla e la stringa da stampare sarà vuota.\\
Inizializzo la \textit{stringa} inserendo le coordinate (date dal \textit{punto}) della piastrella di origine e seguo la sequenza di direzioni, se trovo una \textit{piastrella spenta} (e quindi non esistente) mi fermo e resetto la stringa da stampare.\\ \\
\textbf{Tempo}: \(O(n)\), \(n\) numero di direzioni specificate (sulle quali iterare)\\
\begin{itemize}
  \item \(O(n)\): iterazione sulle \(n\) direzioni
  \item \(O(1)\): funzione \textit{calcolaDeltaVertice} come precedentemente indicato per ogni nuova piastrella in tempo costante
\end{itemize}
\textbf{Spazio}: \(O(n)\), dovuto alle variabili locali utilizzate per le piastrelle visitate e la stringa di output che verrà stampata; questo quindi è pari al numero \(n\) di piastrelle visitate durante il calcolo

\subsection{calcolaPistaBreve}\label{subsec:bfs}
Calcola la \textit{pista più breve} (definita nella sezione "\nameref{sec:pista}"), cioè il cammino minimo tra due vertici \((x_1,\ y_1)\) e \((x_2,\ y_2)\) dati e ne restituisce la \textit{lunghezza}.\\
Viene utilizzata, come indicato nella sezione "\nameref{sec:pista}", una ricerca in ampiezza (\textit{BFS}) trattando il problema come grafo non pesato.\\
La funzione \textit{calcolaPistaBreve}, compresa la \textit{BFS} è stata implementata seguendo questi passaggi:\\
\begin{itemize}
  \item \(O(1)\): verifica che la piastrella di \textit{origine} e \textit{arrivo} siano \textit{accese} altrimenti si ferma la computazione
  \item \(O(1)\): inizializzazione della \textit{mappa visitate} e \textit{coda per i vertici}
  \item \(O(1)\): segno nella mappa la piastrella di origine come visitata
  \item \(O(V\ +\ E)\): fin quando la \textit{coda} non è \textit{vuota}, algoritmo \textit{BFS} (\(V\) vertici, \(E\) archi)
  \begin{itemize}
    \item estraggo il primo vertice della coda
    \item ottengo le piastrelle circonvicine al vertice corrente
    \item per ogni piastrella adiacente, non ancora visitata, la aggiungo alla coda e segno come visitata
    \item incremento la \textit{lunghezza} di quel vertice, come somma della distanza del vertice precedente e \(1\)
    \item controllo se il vertice che sto segnando come visitato e di cui ho incrementato la distanza sia quello di arrivo, in quel caso, fermo la computazione e restituisco il valore della \textit{lunghezza} per quel vertice
    \item se la coda è vuota e non ho trovato il vertice, non esiste una pista e restituisco \(0\) come valore di lunghezza (come da specifica)
  \end{itemize}
\end{itemize}

\textbf{Tempo}: \(O(V\ +\ E)\), con \(V\) numero di piastrelle del piano da esplorare, \(E\) numero di archi (vertici in comune delle piastrelle)\\
\begin{itemize}
  \item \(O(1)\): Inizializzo le strutture dati e inserisco la \textit{piastrella d'origine} nella coda
  \item \(O(1)\): verifico che la piastrella attuale sia quella di \textit{arrivo}, accesso in tempo costante alla mappa 
  \item \(O(V\ +\ E)\): ciclo della \textit{BFS}, quindi per ogni \textit{piastrella circonvicina}
\end{itemize}
La complessità temporale di questa funzione è influenzata quindi dalla \textbf{BFS}.\\
\textbf{Spazio}: \(O(V)\), vertici \(V\) memorizzati (piastrelle) nelle mappe, proporzionalmente al loro numero.
\\ \\
\begin{itemize}
  \item \(O(V)\): la \textit{coda} contiene al massimo \(V\) vertici
  \item \(O(V)\): la mappa \textit{visitate} tiene traccia di \(V\) vertici visitati (piastrelle)
  \item \(O(V)\): la mappa \textit{lunghezza[punto]int} tiene traccia delle distanze \(V\) vertici visitati da quelli precedenti
\end{itemize}

\subsection{calcolaColoriIntorno}\label{subsec:evalColorsAround}
Popola e restituisce una mappa (\textit{map[string]int}) costituita dalle etichette di colore (\textit{string}) e il relativo valore (\textit{int}) nell'intorno di piastrelle.
\\ \\
\textbf{Tempo}: \(O(1)\), \textit{for-range} esegue 8 iterazioni (sono 8 le direzioni, quindi il numero delle piastrelle vicine è costante), mantenendo costante l'accesso alla mappa ed eventualmente aggiungendo un elemento a quest'ultima (in \(O(1)\))\\
\textbf{Spazio}: \(O(1)\), la mappa \textit{coloriIntorno} contiene al massimo 8 elementi distinti


\section{Requisiti e Testing}
Per testare ed utilizzare il programma contenuto all'interno di questo archivio sono sufficienti una versione di \textbf{go} recente presente nel \textit{\$PATH} ed una \textit{shell} dove eseguire alcuni semplici comandi.\\
Se questi requisiti sono soddisfatti, dopo essersi posizionati nella cartella principale (978578\_manca\_kevin), è sufficiente eseguire i seguenti comandi:

\begin{itemize}
  \item{\texttt{go mod init 978578\_manca\_kevin}}: crea un file \textit{go.mod} nella cartella radice del progetto (serve per indicare la posizione dei test e la versione di go)\\
  \item{\texttt{go build}}: che assembla il codice sorgente presente nel \textit{file *.go} contenente la funzione \textit{main}, generando un eseguibile\\ 
  \item{\texttt{go test}}: esegue tutte le funzioni presenti nel file \textbf{*\_test.go} (è possibile aggiungere la flag \texttt{-v} per un risultato più verboso contenente tutti i test eseguiti)\\

\end{itemize}
Un altro possibile metodo è quello di utilizzare l'eseguibile generato dal comando \textit{go build} e reindirizzarli dei file di input o usare il pipe per ottenere il risultato.
\\
Ecco alcuni esempi con i \textit{placeholder} da sostituire
\begin{lstlisting}[language=bash, caption=Comandi shell, basicstyle=\small]
cd "978578_manca_kevin"
go mod init 978578_manca_kevin
go build
# Utilizzando il file eseguibile (passando delle stringhe separate da '\n')
echo "STRINGA_COMANDO" | ./978578_manca_kevin
# Utilizzando il file eseguibile e reindirizzando l'input dei file da testare
./978578_manca_kevin < FILE_INPUT
# Senza l'utilizzo del file eseguibile, direttamente tramite go run
go run 978578_manca_kevin.go < FILE_INPUT
# oppure
echo "STRINGA_COMANDO" | go run ./978578_manca_kevin.go
\end{lstlisting}

\subsection{Test}
Come indicato nella sezione "\nameref{sec:intro}", sono presenti all'interno della cartella \texttt{test/} ulteriori cartelle con i file di \textit{input} e quelli di \textit{output} che ci si aspetta (expected).\\
Tali file mostrano il comportamento del programma relativo a particolarità della specifica, e ciascuno ha il formato \texttt{BaseNomeFunzioni}.\\
Ecco alcuni esempi:\\

\begin{itemize}
  \item  Esempio input \textit{blocco} e \textit{blocco omogeneo} e il loro output\\
  \begin{minipage}[t]{0.45\textwidth}
  \begin{verbatim}
  C 1 1 rosso 2
  C 1 2 rosso 2
  C 1 0 verde 5
  b 1 1
  B 1 1
  B 1 0
  q
  \end{verbatim}
  \end{minipage}
  \hfill
  \begin{minipage}[t]{0.45\textwidth}
  \begin{verbatim}
  9
  4
  4
  \end{verbatim}
  \end{minipage}
  Questo esempio mostra come il calcolo di \textit{blocco} e \textit{blocco omogeneo} applicato nella stessa piastrella(\(1,\ 1\)) restituisce nel primo caso l'intensità pari a quella totale delle piastrelle inizializzate è pari a \(2+2+5 = 9\), mentre nel secondo caso \(2+2 = 4\) in quanto la piastrella (\(1,\ 0\)) non ha lo stesso colore della piastrella sulla quale viene chiamato il blocco.\\

  \item  Esempio input \textit{propaga} e \textit{propagaBlocco} e il loro output\\
  \begin{minipage}[t]{0.45\textwidth}
  \begin{verbatim}
  C 2 2 g 3
  C 2 3 g 2
  C 2 1 r 5
  r v 1 g 1 r
  r m 2 g
  p 2 2
  p 2 3
  P 2 1
  s
  \end{verbatim}
  \end{minipage}
  \hfill
  \begin{minipage}[t]{0.45\textwidth}
  \begin{verbatim}
  (
  v: 1 g 1 r
  m: 2 g
  )
  \end{verbatim}
  \end{minipage}
  Con le piastrelle di coordinate (\(2,\ 2\)), (\(2,\ 3\)), (\(2,\ 0\)), e le regole definite questo esempio mostra come la propagazione del colore tramite \textit{propaga} e \textit{propagaBlocco}.\\ 
  La propagazione sulla piastrella (\(2,\ 2\)) comporta che la prima regola venga applicata (\textit{r v 1 g 1 r}) ricolorando la piastrella di \texttt{v}.\\
  La propagazione sulla piastrella (\(2,\ 1\)) comporta che la seconda regola venga applicata (\textit{r e 2 g}) ricolorando la piastrella di \texttt{m}.\\
  Effettuando infine \textit{propagaBlocco} sulla piastrella (\(2,\ 1\)) le piastrelle risultano:\\
  \begin{itemize}
    \item Piastrella(2, 2): v 3
    \item Piastrella(2, 3): g 2
    \item Piastrella(2, 1): r 5
  \end{itemize}

\item Esempio input \textit{pista} e \textit{lung} e il loro output\\
  \begin{minipage}[t]{0.45\textwidth}
  \begin{verbatim}
  C 3 0 r 8
  C 2 1 h 4
  C 2 2 i 8
  t 3 0 NO,NN
  L 3 0 2 1
  q
  \end{verbatim}
  \end{minipage}
  \hfill
  \begin{minipage}[t]{0.45\textwidth}
  % \begin{itemize}
  \begin{verbatim}
  [
  3 0 r 8
  2 1 h 4
  2 2 i 8
  ]
  2
  \end{verbatim}
  % \end{itemize}
  \end{minipage}
  Con le piastrelle di coordinate (\(3,\ 0\)), (\(2,\ 1\)), (\(2,\ 2\)), viene mostrato nell'esempio la \textit{pista} risultante e la sua \textit{lunghezza}.\\ 
  La \textit{pista}, seguendo le \textit{direzioni NO,NN} partendo dalla piastrella(\(3,\ 0\)) risulta appunto \((3,\ 0) \rightarrow\ (2,\ 1)\ \rightarrow\ (2,\ 2)\).\\
  Mentre la \textit{lunghezza} partendo dalla piastrella(\(3,\ 0\)) e arrivando alla piastrella(\((2,\ 1\)) risulta pari a \(2\).

\end{itemize}
  
\end{document}

